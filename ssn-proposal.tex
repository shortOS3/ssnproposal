\documentclass{article}
\usepackage[affil-it]{authblk}

\renewcommand{\familydefault}{\sfdefault}

%\author{Alexandros Stavroulakis, Alexandros.Stavroulakis@os3.nl\\Xavier Torrent Gorj\'{o}n, Xavier.TorrentGorjon@os3.nl
%\\Nikolaos Petros Triantafyllidis, Nikolaos.Triantafyllidis@os3.nl}


\author{Alexandros Stavroulakis%
  %\thanks{Electronic address: \texttt{Alexandros.Stavroulakis@os3.nl}}}
%\affil{University of Amsterdam, System and Network Engineering (MSc
}

\author{Xavier Torrent Gorj\'{o}n%
%  \thanks{Electronic address: \texttt{Xavier.TorrentGorjon@os3.nl}}}
%\affil{University of Amsterdam, System and Network Engineering (MSc
}

\author{Nikolaos Petros Triantafyllidis%
  \thanks{e-mail: Alexandros.Stavroulakis@os3.nl, Xavier.TorrentGorjon@os3.nl, Nikolaos.Triantafyllidis@os3.nl}}
\affil{University of Amsterdam, System and Network Engineering (MSc)}

%\author{Xavier Torrent Gorj\'{o}n
%	\thanks{Electronic address: \texttt{Nikolaos.Triantafyllidis@os3.nl, }; }
%\affil{University of Amsterdam, System and Network Engineering (MSc)}

%\author{Nikolaos Petros Triantafyllidis
%  \thanks{Electronic address: \texttt{Nikolaos.Triantafyllidis@os3.nl, }; }
%\affil{University of Amsterdam, System and Network Engineering (MSc)}


\title{ Project Proposal \\ Security of Systems and Networks\\ Exhaustive search on URL shorteners}




\begin{document}

\maketitle


\section{Introduction}
As web resources keep growing in complexity, so does the size of the URLs used to access specific web content. These URLs can contain certain parameters regarding user preferences, search queries or the structure of a website, which if exposed might lead to potential security issues. A new type of service has appeared on the Internet in the last few years, the URL shorteners (goo.gl, bitly.com,  tinyurl.com, among others). These services provide means to share these addresses on platforms with limited size (SMS,Twitter, etc.). \\
We believe that these services can potentially expose unaware users to security vulnerabilities. Even if these shortened URLs are shared through private channels, they can still be easily guessed through brute force attacks trying sequential combinations of characters to synthesise those shortened URLs (since the information is public) and retrieve the actual URLs behind them. Although we believe this cannot be a targeted attack, a malicious user could still retrieve vast amounts data from many Internet users and detect possible attacks to be performed at a later stage. 


\section{Research Questions}

The questions we aim to answer in this project are the following:
\begin{itemize}
  \item What kind of data can be retrieved from such an attack? 
  \item Is this information enough to be used as an entry point for other attacks?
  \item Is there any way to narrow the dataset in order to focus the attack? (!!!!!! check again !!!!!) 
\end{itemize}
This project focuses on the most vulnerable side of computer security: the users themselves. Even though we will try to determine if there are vulnerabilities on the URL shorteners themselves, most of the information leakage will come from users unaware of this kind of attacks.

\section{Approach}
We will develop a URL crawler for these websites that will retrieve data from a certain amount of shortened URLs, the only constraint being the deadlines for this project. After enough data has been gathered, and a solid URL database has been built, we will try to extract sensitive information from the full-sized retrieved URLs by applying data mining and pattern matching techniques. \\
Software-wise, we will use an appropriate web programming language (probably Python or Go based on how they perform) to build the data aggregating software. The services will be deployed on our assigned SNE servers.\\
Depending on the amount of data gathered and the time remaining Big Data analysis techniques might need to be employed as well. 

\section{Planning}
The project officially starts on November 17th and its final presentation is on December 15th, resulting in four working weeks. The planning for these weeks is as follows:
\\\\
\begin{table}[h]
  \label{tbl:example}
  \begin{tabular}{|c|c|}
    \hline
    Week  & Work \\
    \hline
    November 17th & Background research \\
    \hline
    November 17th & Literature review   \\
    \hline
    November 24th & Software development \\
    \hline
    November 24th & Bulk data aggregation \\
    \hline
    December 1st & Start designing the data mining techniques\\
    \hline
    December 1st & Data aggregation finishes\\
    \hline
    December 8th & Application of pattern search on full dataset \\
    \hline
    December 8th & Analysis of results \\
    \hline
    December 8th & Report composition  \\
    \hline
    December 15th & Presentation \\
    \hline
  \end{tabular}
\end{table}

\section{Ethical implications}
All the information we will try to retrieve is publicly available on the Internet. However, we might encounter sensitive information (such as user preferences, birth dates, network configurations, etc.) in the process. That is the reason we will have to develop our crawlers and storage system in a way that the information we gather will not be leaked.

\end{document}
